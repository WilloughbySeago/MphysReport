% !TeX program = lualatex
\documentclass[fleqn]{NotesClass}

\strictpagecheck

%\usepackage{ebgaramond}

%% Packages
\usepackage{csquotes}
\usepackage{tensor}
\usepackage{ytableau}
\ytableausetup{centertableaux}

% Tikz stuff
\usepackage{tikz}
% External
\usetikzlibrary{external}
\tikzexternalize[prefix=tikz-external/]
% Other libraries

\usepackage{tikz-cd}

\RequirePackage[%
sorting=none,  % Don't sort the references, they will appear in the order they are first cited
style=numeric-comp,  % citations like [1] and citing 1, 2, and 3 gives [1-3]
giveninits=true,  % style author names as J. Doe
language=british  % Make dates dd/mm/yyyy
]{biblatex}
\addbibresource{ref.bib}
\usepackage{xurl}

% References, should be last things loaded
\usepackage[pdfauthor={Willoughby Seago},pdftitle={MPhys Report: Computational Group Theory},pdfkeywords={group theory, representation theory, birdtracks, Lie theory},pdfsubject={Lie Groups, Representation Theory}]{hyperref}  % Should be loaded second last (cleveref last)
\colorlet{hyperrefcolor}{blue!60!black}
\hypersetup{colorlinks=true, linkcolor=hyperrefcolor, urlcolor=hyperrefcolor}
\usepackage[
capitalize,
nameinlink,
noabbrev
]{cleveref} % Should be loaded last

% My packages
\usepackage{NotesBoxes}
\usepackage{NotesMaths}

\setmathfont[range={\int, \oint, \otimes, \oplus, \bigotimes, \bigoplus}]{Latin Modern Math}

% Highlight colour
\definecolor{highlight}{HTML}{710D78}
\definecolor{my blue}{HTML}{2A0D77}
\definecolor{my red}{HTML}{770D38}
\definecolor{my green}{HTML}{14770D}
\definecolor{my yellow}{HTML}{E7BB41}
\colorlet{light highlight}{highlight!30}
\colorlet{example background}{azure(web)(azuremist)!45}
\colorlet{background color}{white}

\AtBeginEnvironment{exm}{\colorlet{background color}{example background}}
\AtEndEnvironment{exm}{\colorlet{background color}{white}}

% Title page info
\title{Computational Group Theory}
\author{Willoughby Seago}
\date{\today}
\subtitle{MPhys Project Report}
\subsubtitle{Supervised by Tony Kennedy}

% Commands
% Tikz
\tikzset{wire/.style={thick}}
\tikzset{underwire/.style={line width=1mm, background color}}
\tikzset{symmetriser/.style={fill=background color, thick, rounded corners=1pt}}
\tikzset{antisymmetriser/.style={fill=black, thick, rounded corners=1pt}}

% Text

% Maths
\newcommand{\identity}{1}
\newcommand{\identityMatrix}{\symbb{I}}
\newcommand{\symmetricGroup}[1][n]{S_{#1}}
\DeclareMathOperator{\Mat}{Mat}
\RenewDocumentCommand{\matrices}{ o m m }{
    \IfNoValueTF{#1}{
        \Mat(#3, #2)
    }{
        \Mat(#3, #1 \times #2)
    }
}
\renewcommand{\field}{\symbb{k}}
\newcommand{\trans}{\top}
\newcommand{\hermit}{\dagger}
\newcommand{\action}{\mathbin{.}}
\ExplSyntaxOn
% Create LaTeX interface command
\NewDocumentCommand{\cycle}{ O{\,} m }{  % optional arg is separator, mandatory
    %arg is comma separated list
    (
    \willoughby_cycle:nn { #1 } { #2 }
    )
}

\clist_new:N \l_willougbhy_cycle_clist  % Create new clist variable
\cs_new_protected:Npn \willoughby_cycle:nn #1 #2 {  % create LaTeX3 function
    \clist_set:Nn \l_willougbhy_cycle_clist { #2 }  % set clist variable with
    %clist #2 passed by user
    \clist_use:Nn \l_willougbhy_cycle_clist { #1 }  % print list separated by #1
}
\ExplSyntaxOff
\DeclarePairedDelimiter{\tuple}{\langle}{\rangle}
\newcommand{\isomorphic}{\cong}
\newcommand{\projector}[1]{\symbfup{P}_{#1}}
\newcommand{\hooknumber}[1]{\abs{#1}}
\newcommand{\dual}[1]{{#1^{*}}}
\newcommand{\lorentzGroup}{\specialOrthogonal^+(1, 3)}
\newcommand{\minkowskiSpace}{\reals^{1,3}}


\includeonly{}

\begin{document}
    \frontmatter
    \titlepage
    \maketitle
    \tableofcontents
    % \listoffigures
    % \listoftables
    \mainmatter
    
    \chapter{Mathematical Preliminaries}
    Physics is full of tensors, they appear in every field, from the moment of inertia tensor in classical mechanics to observables in quantum mechanics, from the electromagnetic field strength in electrodynamics to the curvature tensor in general relativity.
    As such being able to quickly and efficiently manipulate and perform computations with tensors is of utmost importance to all physicists.
    Unfortunately the classic physicist definition of a tensor is the famously unhelpful
    \begin{displayquote}
        A tensor is something which transforms like a tensor.
    \end{displayquote}
    Usually this is then followed by a definition of how a tensor transforms in a given setting, say rotations in classical mechanics or Lorentz transformations in relativity.
    We will instead start with a more general definition of a tensor, but for this we will require some more mathematics first.
    
    
    \section{Groups}
    Groups capture the idea of a symmetry in a precise and mathematical way.
    Intuitively a symmetry is something we can do to a system which leaves the system unchanged, or \defineindex{invariant}, under that symmetry.
    We can abstract the notion of a symmetry through four requirements.
    Given some collection of symmetries there must be a way to combine them, do nothing, and undo any of the symmetries, and the final requirement is that the way we use brackets doesn't matter.
    This leads to the following definition.
    
    \begin{dfn}{Group}{}
        A \defineindex{group}, \((G, \cdot)\), is a set, \(G\), and a binary operation, \(\cdot \colon G \times G \to G\) such that the following axioms are satisfied \cite{riley-hobson-bence}:
        \begin{description}
            \item[Identity]\index{identity} there exists some distinguished element, \(\identity \in G\), such that for all \(x \in G\) we have \(\identity \cdot x = x \cdot \identity = x\);
            \item[Inverse]\index{inverse} for all \(x \in G\) there exists some \(x^{-1} \in G\) such that \(x \cdot x^{-1} = x^{-1} \cdot x = \identity\);
            \item[Associativity]\index{associativity} for all \(x, y, z \in G\) we have \((x \cdot y) \cdot z = x \cdot (y \cdot z)\).
        \end{description}
    \end{dfn}
    
    We follow the common abuse of terminology and refer to \(G\) alone as the group with the operation left implicit.
    We will also write most group operations as juxtaposition, writing \(xy\) for \(x \cdot y\).
    
    The prototype for a group is the \defineindex{symmetric group} on \(n\) objects, \(\symmetricGroup\) \cite{james-rep-symmetric-group}.
    This is defined as the set
    \begin{equation}
        \symmetricGroup \coloneqq \{ \sigma \colon \{1, \dotsc, n\} \to \{1, \dotsc, n\} \mid \sigma \text{ is a bijection} \}
    \end{equation}
    with function composition as the group operation.
    We will use cycle notation for elements of \(\symmetricGroup\), where a cycle sends each element to the next in the list and the last element in the list to the first.
    For example, \(\cycle{1,3,4}\) sends \(1\) to \(3\), \(3\) to \(4\), and \(4\) to \(1\).
    In this notation the identity is the empty cycle, \(\cycle{}\).
    
    Another important family of groups are various collections of matrices with matrix multiplication as the group operation.
    In this case the group identity is the identity matrix of  the appropriate dimension, \(\identityMatrix\).
    Let \(\matrices{n}{\field}\) denote the set of \(n \times n\) matrices with entries in \(\field\).
    The following are all groups under matrix multiplication \cite{allanach}:
    \begin{itemize}
        \item \defineindex{general linear group} \(\generalLinear(n, \field) \coloneqq \{M \in \matrices{n}{\field} \mid M \text{ is invertible} \}\);
        \item \defineindex{special linear group} \(\specialLinear(n, \field) \coloneqq \{M \in \generalLinear(n, \field) \mid \det M = 1\}\);
        \item \defineindex{orthogonal group} \(\orthogonal(n) \coloneqq \{O \in \matrices{n}{\reals} \mid O^\trans O = \identityMatrix \}\);
        \item \defineindex{special orthogonal group} \(\specialOrthogonal(n) \coloneqq \{O \in \orthogonal(n) \mid \det O = 1\}\);
        \item \defineindex{unitary group} \(\unitary(n) \coloneqq \{U \in \matrices{n}{\complex} \mid U^\hermit U = \identityMatrix \}\);
        \item \defineindex{special unitary group} \(\specialUnitary(n) \coloneqq \{U \in \unitary(n) \mid \det U = 1\}\).
    \end{itemize}
    These matrix groups are actually examples of a Lie group:
    \begin{dfn}{Lie Group}{}
        A \defineindex{Lie group}, \(G\), is a group which is also a manifold in a compatible way \cite{san-martin-lie-groups}.
        By this we mean that the product map \(\cdot \colon G \times G \to G\) is analytic.
    \end{dfn}
    
    Most of the time when considering a group we think of the symmetries it represents being applied to some object.
    This leads to the following definition.
    
    \begin{dfn}{Group Action}{}
        Let \(G\) be a group and \(X\) a set.
        A (left) \defineindex{group action} of \(G\) on \(X\) is a function \(\varphi \colon G \times X \to X\) such that \cite[713]{hassani}
        \begin{description}
            \item[Identity] for all \(x \in X\) we have \(\varphi(\identity, x) = x\),
            \item[Compatibility] for all \(g, h \in G\) and \(x \in X\) we have \(\varphi(g, \varphi(h, x)) = \varphi(gh, x)\) where \(gh\) is the product of \(g\) and \(h\) in \(G\).
        \end{description}
        We usually write \(\varphi(g, x) = g \action x\) or even \(\varphi(g, x) = gx\).
        In this case we have \(\identity \action x = x\) and \(g \action (h \action x) = (gh) \action x\), with \(gh\) being the product of \(g\) and \(h\) computed as elements of \(G\).
    \end{dfn}
    
    The symmetric group, \(\symmetricGroup\), acts on \(n\)-tuples, \(\tuple{a_1, \dotsc, a_n}\), by permuting the elements.
    For example, \(\cycle{1,3,4} \in \symmetricGroup[5]\) acts on \(\tuple{a_1, a_2, a_3, a_4, a_5}\) by
    \begin{equation}
        \cycle{1,3,4} \action \tuple{a_1, a_2, a_3, a_4, a_5} = \tuple{a_4, a_2, a_1, a_3, a_5}.
    \end{equation}
    Note that this action is done by permuting the symbols \(a_i\), rather than by permuting the values of \(i\), which would instead give \(\tuple{a_3, a_2, a_4, a_1, a_5}\).
    
    Matrix groups, such as \(\generalLinear(n, \field)\), have a natural action on the \(n\)-dimensional vector space \(\field^n\) by interpreting elements of \(\field^n\) as column vectors and then acting on them through matrix multiplication.
    The action of matrices on vector spaces in this form turns out to be a very useful way of thinking about a group, since matrix multiplication is simple and easy to perform on a computer.
    This insight leads to the idea of a representation, the subject of the next section.
    
    \begin{dfn}{Morphisms}{}
        Let \(G\) and \(H\) be groups.
        A \defineindex{group homomorphism} is a map \(\varphi \colon G \to H\) such that \(\varphi(gg') = \varphi(g)\varphi(g')\) for all \(g, g' \in G\) \cite[705]{hassani}.
        If \(\varphi\) is invertible then we call it an \defineindex{isomorphism}.
        If there is an isomorphism between \(G\) and \(H\) we say that \(G\) and \(H\) are \defineindex{isomorphic}, and denote this \(G \isomorphic H\).
    \end{dfn}
    
    Notice that if \(\identity_G\) and \(\identity_H\) are the identity elements of \(G\) and \(H\) respectively then we have \(\varphi(\identity_G) = \identity_H\) as well as \(\varphi(g^{-1}) = \varphi(g)^{-1}\) for all \(g \in G\).
    
    \begin{dfn}{Group Algebra}{}\index{group algebra}
        Let \(G\) be a group and \(R\) a ring.
        The group ring, \(R[G]\), is the set of formal sums
        \begin{equation}
            \sum_{g \in G} r_g g
        \end{equation}
        where \(r_g \in R\) is zero for all but a finite number of elements \(g\).
        This set of formal sums is a ring with addition defined as
        \begin{equation}
            \sum_{g \in G} r_g g + \sum_{g \in G} s_g g \coloneqq \sum_{g\in G}(r_g + s_g) g
        \end{equation}
        and multiplication defined as
        \begin{equation}
            \left( \sum_{g \in G} r_g g \right) \left( \sum_{h \in G} s_h h \right) \coloneqq \sum_{g, h \in G}^{g} r_g s_h gh
        \end{equation}
        with the product \(r_gs_h\) computed in \(R\) and the product \(gh\) computed in \(G\).
        If \(R\) is a field then \(R[G]\) is an associative algebra (a vector space with an associative product), called the \defineindex{group algebra} \cite[740]{hassani}.
    \end{dfn}
    
    The most common case of a group algebra is the group algebra \(\complex[G]\) for some arbitrary group, \(G\).
    
    \section{Representations}
    \begin{dfn}{Representation}{}
        Let \(G\) be a group.
        A \defineindex{group representation}, \((\rho, V)\), is a pair consisting of a vector space, \(V\), called the representation space, and a homomorphism \(\rho \colon G \to \generalLinear(V)\) \cite[726]{hassani}.
        Here \(\generalLinear(V) \coloneqq \{T \colon V \to V \mid T \text{ is linear}\}\) is the group of automorphisms of \(V\) with function composition as the group operation.
        Fixing some basis for \(V\) we can identify \(\generalLinear(V) \isomorphic \generalLinear(\dim V, \field)\) if \(V\) is a \(\field\)-vector space.
    \end{dfn}
    
    Another way of defining a representation is as a group action of \(G\) on \(V\) given by \(g \action v = \rho(g)v\) for \(g \in G\) and \(v \in V\).
    
    It is common to refer to both \(\rho\), \(V\), alone, as well as the pair \((\rho, V)\) as the representation.
    The simplest example of a representation is the \defineindex{trivial representation}, \((\rho_{\text{trivial}}, V)\), which acts trivially on \(V\), that is \(\rho_{\text{trivial}}(g) = \identityMatrix\) for all \(g \in G\) and so \(g \action v = v\) for all \(g \in G\).
    
    \begin{exm}{Permutation Representation}{}
        Consider the symmetric group on three elements, \(\symmetricGroup[3]\).
        This has a representation on \(\reals^3\) given by identifying
        \begin{equation*}
            \rho\cycle{1,2} = 
            \begin{pmatrix}
                0 & 1 & 0\\
                1 & 0 & 0\\
                0 & 0 & 1
            \end{pmatrix}
            ,\ \rho\cycle{1,3} = 
            \begin{pmatrix}
                0 & 0 & 1\\
                0 & 1 & 0\\
                1 & 0 & 0
            \end{pmatrix}
            , \ \text{and} \  \rho\cycle{2,3} = 
            \begin{pmatrix}
                1 & 0 & 0\\
                0 & 0 & 1\\
                0 & 1 & 0
            \end{pmatrix}
            .
        \end{equation*}
        This then acts by permuting the basis vectors \(\ve{1} = (1, 0, 0)^\trans\), \(\ve{2} = (0, 1, 0)^\trans\), and \(\ve{3} = (0, 0, 1)^\trans\), so we call this the \defineindex{permutation representation}.
        The representation of any other group element can be found by writing the element as a product of \define{transpositions}\index{transposition} (two element cycles).
        For example, \(\cycle{1,2,3} = \cycle{1,2}\cycle{2,3}\) and so
        \begin{multline}
            \rho\cycle{1,2,3} = \rho(\cycle{1,2} \cycle{2,3}) = \rho\cycle{1,2} \, \rho\cycle{2,3}\\
            = 
            \begin{pmatrix}
                0 & 1 & 0\\
                1 & 0 & 0\\
                0 & 0 & 1
            \end{pmatrix}
            \begin{pmatrix}
                1 & 0 & 0\\
                0 & 0 & 1\\
                0 & 1 & 0
            \end{pmatrix}
            = 
            \begin{pmatrix}
                0 & 0 & 1\\
                1 & 0 & 0\\
                0 & 1 & 0
            \end{pmatrix}
            .
        \end{multline}
    \end{exm}
    
    There are an infinite number of representations of any group, given some representation \((\rho, V)\) we can always consider some larger space \(W \supset V\) and define \(\rho' \colon G \to \generalLinear(W)\) so that \(\rho'(g)\) acts on the subspace \(V\) as \(\rho(g)\).
    Clearly this representation doesn't really give us any new information.
    For this reason we define irreducible representations.
    
    \begin{dfn}{Irreducible Representation}{}
        Let \(G\) be a group and \((\rho, V)\) a representation of \(G\).
        We say that \((\rho, V)\) is an \defineindex{irreducible representation}, or irrep, if \(V\) has no \(G\)-invariant subspaces \cite{hamermesh}.
        That is, there is no \(W \subset V\) such that \(g \action w \in W\) for all \(w \in W\).
    \end{dfn}

    \begin{dfn}{Decomposable Representation}{}
        Let \(G\) be a group and \((\rho, V)\) a representation of \(G\).
        We say that \((\rho, V)\) is a \defineindex{decomposable representation} if \(\{\rho(g)\}\) can be simultaneously diagonalised \cite{hamermesh}.
        In other words, there exist representations \((\rho_i, V_i)\) such that \(\rho = \bigoplus_i \rho_i\) and \(V = \bigoplus_i V_i\).
    \end{dfn}
    
    For a finite group, \(G\), and a representation space over \(\reals\) or \(\complex\) all indecomposable representations are irreducible \cite{hamermesh}.
    This also the case if \(G\) is compact.
    We will assume that all indecomposable are irreducible and vice versa.
    
    We are now in a position to make the physicist definition of a tensor more  precise.
    \begin{dfn}{Tensor}{}
        Let \(G\) be a group and \(\rho \colon G \to \generalLinear(V)\),
        and denote by \(\dual{V}\) the dual space to \(V\).
        A \defineindex{tensor}, \(T \in V^p \otimes \dual{V}^q \), is an object with components \(\tensor{T}{^{a_1\dotso a_q}_{b_1 \dotso b_p}}\), which transform under the action of \(g \in G\) according to
        \begin{equation*}
            \tensor{{T'}}{^{a_1\dotso a_q}_{b_1 \dotso b_p}} = \tensor{\rho(g)}{^{a_1}_{c_1}} \dotsm \tensor{\rho(g)}{^{a_q}_{c_q}} \tensor{\rho(g)}{_{b_1}^{d_q}} \dotsm \tensor{\rho(g)}{_{b_p}^{d_p}} \tensor{T}{^{c_1\dotso c_q}_{d_1 \dotso d_p}}
        \end{equation*}
        where we employ the Einstein summation convention and sum over repeated indices \cite[18]{cvitanovic}.
    \end{dfn}
    \begin{exm}{}{}
        In classical mechanics we consider Cartesian tensors, which transform under the rotation group \(\specialOrthogonal(3)\).
        For example, if \(R \in \specialOrthogonal(3)\) is a rotation matrix then \(\vv{x} \in \reals^3\) is a vector if it has components transforming according to
        \begin{equation}
            x_i = R_{ij}x_j.
        \end{equation}
        The moment of inertia tensor, \(I\), has components \(I_{ij}\) and transforms as
        \begin{equation}
            I'_{ij} = R_{ik}R_{jl}I_{kl}.
        \end{equation}
        
        In relativity we consider Lorentz tensors, which transform under the Lorentz group \(\lorentzGroup\).
        For example, if \(\Lambda \in \lorentzGroup\) is a Lorentz transformation then \(x \in \minkowskiSpace\) is a vector if it has components transforming according to
        \begin{equation}
            x^\mu = \tensor{\Lambda}{^\mu_\nu}x^\nu.
        \end{equation}
        The energy-momentum tensor, \(T\), has components \(T^{\mu\nu}\) which transform as
        \begin{equation}
            T'^{\mu\nu} = \tensor{\Lambda}{^\mu_\rho}\tensor{\Lambda}{^\nu_\sigma} T^{\rho\sigma}.
        \end{equation}
    \end{exm}
    
    \section{Birdtracks}
    A permutation in \(\symmetricGroup\) can pictured with a \defineindex{braid diagram}, which tracks \(n\) elements swapping through wires connecting inputs and outputs \cite[49]{cvitanovic}.
    For example, the permutation \(\cycle{1,2,4} \in \symmetricGroup[5]\) can be drawn, including labels normally left implicit, as
    \begin{equation}
        \tikzsetnextfilename{math-prelims-example-braid}
        \begin{tikzpicture}[baseline=(current bounding box)]
            \draw[wire] (0, 0.5) node [left] {5} -- (2.5, 0.5) node [right] {5};
            \draw[wire] (0, 1.5) node [left] {4} -- (2.5, 1.5) node [right] {4};
            \draw[underwire, rounded corners] (0, 1) -- (0.5, 1) -- (2, 2.5) -- (2.5, 2.5);
            \draw[wire, rounded corners] (0, 1) node [left] {3} -- (0.5, 1) -- (2, 2.5) -- (2.5, 2.5) node [right] {3};
            \draw[underwire, rounded corners] (0, 2.5) -- (0.5, 2.5) -- (1, 2) -- (2.5, 2);
            \draw[wire, rounded corners] (0, 2.5) node [left] {1} -- (0.5, 2.5) -- (1, 2) -- (2.5, 2) node [right] {1};
            \draw[underwire, rounded corners] (0, 2) -- (0.5, 2) -- (1.5, 1) -- (2.5, 1);
            \draw[wire, rounded corners] (0, 2) node [left] {2} -- (0.5, 2) -- (1.5, 1) -- (2.5, 1) node [right] {2};
        \end{tikzpicture}
        \ .
    \end{equation}
    
    In this notation two permutations can be composed by writing the diagrams \emph{in the opposite order to the product} and joining up the inputs of one to the outputs of the other.
    For example, the product \(\cycle{1,2,4} \cycle{3,4} = \cycle{1,2,4,3}\), viewed in \(\symmetricGroup[5]\), can be computed by connecting up the relevant diagrams:
    \begin{equation}
        \tikzsetnextfilename{math-prelims-example-diagram-composition}
        \begin{tikzpicture}[baseline=(current bounding box)]
            \draw[wire] (0, 0) -- (2.5, 0);
            \draw[wire] (0, 1) -- (2.5, 1);
            \draw[underwire, rounded corners] (0, 0.5) -- (0.5, 0.5) -- (2, 2) -- (2.5, 2);
            \draw[wire, rounded corners] (0, 0.5) -- (0.5, 0.5) -- (2, 2) -- (2.5, 2);
            \draw[underwire, rounded corners] (0, 2) -- (0.5, 2) -- (1, 1.5) -- (2.5, 1.5);
            \draw[wire, rounded corners] (0, 2) -- (0.5, 2) -- (1, 1.5) -- (2.5, 1.5);
            \draw[underwire, rounded corners] (0, 1.5) -- (0.5, 1.5) -- (1.5, 0.5) -- (2.5, 0.5);
            \draw[wire, rounded corners] (0, 1.5) -- (0.5, 1.5) -- (1.5, 0.5) -- (2.5, 0.5);
            \begin{scope}[xshift=3.5cm]
                \draw[wire] (0, 0) -- (1.5, 0);
                \draw[underwire, rounded corners] (0, 0.5) -- (0.5, 0.5) -- (1, 1) -- (1.5, 1);
                \draw[wire, rounded corners] (0, 0.5) -- (0.5, 0.5) -- (1, 1) -- (1.5, 1);
                \draw[wire, rounded corners] (0, 1) -- (0.5, 1) -- (1, 0.5) -- (1.5, 0.5);
                \draw[wire] (0, 1.5) -- (1.5, 1.5);
                \draw[wire] (0, 2) -- (1.5, 2);
            \end{scope}
            \node at (3, 1) {\(\circ\)};
            \node at (5.5, 1) {\(=\)};
            \begin{scope}[xshift=7.5cm]
                \draw[wire] (0, 0) -- (2.5, 0);
                \draw[wire] (0, 1) -- (2.5, 1);
                \draw[underwire, rounded corners] (0, 0.5) -- (0.5, 0.5) -- (2, 2) -- (2.5, 2);
                \draw[wire, rounded corners] (0, 0.5) -- (0.5, 0.5) -- (2, 2) -- (2.5, 2);
                \draw[underwire, rounded corners] (0, 2) -- (0.5, 2) -- (1, 1.5) -- (2.5, 1.5);
                \draw[wire, rounded corners] (0, 2) -- (0.5, 2) -- (1, 1.5) -- (2.5, 1.5);
                \draw[underwire, rounded corners] (0, 1.5) -- (0.5, 1.5) -- (1.5, 0.5) -- (2.5, 0.5);
                \draw[wire, rounded corners] (0, 1.5) -- (0.5, 1.5) -- (1.5, 0.5) -- (2.5, 0.5);
            \end{scope}
            \begin{scope}[xshift=6cm]
                \draw[wire] (0, 0) -- (1.6, 0);
                \draw[underwire, rounded corners] (0, 0.5) -- (0.5, 0.5) -- (1, 1) -- (1.6, 1);
                \draw[wire, rounded corners] (0, 0.5) -- (0.5, 0.5) -- (1, 1) -- (1.6, 1);
                \draw[wire, rounded corners] (0, 1) -- (0.5, 1) -- (1, 0.5) -- (1.6, 0.5);
                \draw[wire] (0, 1.5) -- (1.6, 1.5);
                \draw[wire] (0, 2) -- (1.6, 2);
            \end{scope}
        \end{tikzpicture}
        \ .
    \end{equation}
    We can then simplify the diagram by assuming that the wires can pass through each other and rearrange them until the diagram is more readable.
    More formally two diagrams represent the same permutation if they are equivalent up to a four-dimensional spatial isotopy, the fourth dimension allowing us to pass the wires around each other, when in three dimensions they would collide.
    This results in the following diagram:
    \begin{equation}
        \tikzsetnextfilename{math-prelims-example-diagram-simplified-composition}
        \begin{tikzpicture}[baseline=(current bounding box)]
            \draw[wire] (0, 0) -- (2, 0);
            \draw[underwire, rounded corners] (0, 0.5) -- (1, 0.5) -- (1.5, 1) -- (2, 1);
            \draw[wire, rounded corners] (0, 0.5) -- (1, 0.5) -- (1.5, 1) -- (2, 1);
            \draw[underwire, rounded corners] (0, 1) -- (0.5, 1) -- (1.5, 2) -- (2, 2);
            \draw[wire, rounded corners] (0, 1) -- (0.5, 1) -- (1.5, 2) -- (2, 2);
            \draw[underwire, rounded corners] (0, 1.5) -- (0.5, 1.5) -- (1.5, 0.5) -- (2, 0.5);
            \draw[wire, rounded corners] (0, 1.5) -- (0.5, 1.5) -- (1.5, 0.5) -- (2, 0.5);
            \draw[underwire, rounded corners] (0, 2) -- (1, 2) -- (1.5, 1.5) -- (2, 1.5);
            \draw[wire, rounded corners] (0, 2) -- (1, 2) -- (1.5, 1.5) -- (2, 1.5);
        \end{tikzpicture}
        \ .
    \end{equation}
    
    A common operation on tensors is to symmetrise or antisymmetrise over a certain set of indices \cite[50--51]{cvitanovic}.
    Denoting by \(S_{i_1\dotso i_k}\) the symmetriser over the indices \(i_1, \dotsc, i_k\) this symmetriser is given by
    \begin{equation}
        S_{i_1 \dotso i_k} \coloneqq \frac{1}{k!} \sum_{\sigma \in \symmetricGroup[k]} \sigma,
    \end{equation}
    where the permutations act on the \(k\) indices \(i_1, \dotsc, i_k\).
    This formal sum of permutations is an element of the group algebra, \(\complex[\symmetricGroup[k]]\).
    Similarly, the antisymmetriser is defined as
    \begin{equation}
        A_{i_1 \dotso i_k} \coloneqq \frac{1}{k!} \sum_{\sigma \in \symmetricGroup[k]} \sgn(\sigma) \sigma
    \end{equation}
    where \(\sgn\) is the sign function, defined to be 1 if \(\sigma\) can be decomposed as an even number of transpositions, and \(-1\) otherwise.
    
    For example, \(S_{12} = (\cycle{} + \cycle{1,2})/2\) and \(A_{12} = (\cycle{} + \cycle{1,2})/2\).
    Acting on a two index tensor, \(T^{ij}\), with these gives
    \begin{align}
        S_{12} \action T^{ij} &= T^{(ij)} = \frac{1}{2}(T^{ij} + T^{ji}),\\
        A_{12} \action T^{ij} &= T^{[ij]} = \frac{1}{2}(T^{ij} - T^{ji}).
    \end{align}
    
    In the braid notation we write a symmetriser as an empty box into which the wires being symmetrised are fed, and the antisymmetriser as a filled in box, so
    \begin{align}
        S_{12} &= 
        \tikzsetnextfilename{math-prelims-example-symmetriser}
        \begin{tikzpicture}[baseline=(equal.base)]
            \draw[wire] (0, 0) -- (1.5, 0);
            \draw[wire] (0, 0.5) -- (1.5, 0.5);
            \draw[symmetriser] (0.5, -0.25) rectangle (1, 0.75);
            \node (equal) at (2, 0.25) {\(=\)};
            \draw[wire] (2.5, 0) -- (4, 0);
            \draw[wire] (2.5, 0.5) -- (4, 0.5);
            \node at (4.5, 0.25) {\(+\)};
            \draw[wire, rounded corners] (5, 0) -- (5.5, 0) -- (6, 0.5) -- (6.5, 0.5);
            \draw[underwire, rounded corners] (5, 0.5) -- (5.5, 0.5) -- (6, 0) -- (6.5, 0);
            \draw[wire, rounded corners] (5, 0.5) -- (5.5, 0.5) -- (6, 0) -- (6.5, 0);
        \end{tikzpicture}
        ,\\
        A_{12} &= 
        \tikzsetnextfilename{math-prelims-example-antisymmetriser}
        \begin{tikzpicture}[baseline=(equal.base)]
            \draw[wire] (0, 0) -- (1.5, 0);
            \draw[wire] (0, 0.5) -- (1.5, 0.5);
            \draw[antisymmetriser] (0.5, -0.25) rectangle (1, 0.75);
            \node (equal) at (2, 0.25) {\(=\)};
            \draw[wire] (2.5, 0) -- (4, 0);
            \draw[wire] (2.5, 0.5) -- (4, 0.5);
            \node at (4.5, 0.25) {\(-\)};
            \draw[wire, rounded corners] (5, 0) -- (5.5, 0) -- (6, 0.5) -- (6.5, 0.5);
            \draw[underwire, rounded corners] (5, 0.5) -- (5.5, 0.5) -- (6, 0) -- (6.5, 0);
            \draw[wire, rounded corners] (5, 0.5) -- (5.5, 0.5) -- (6, 0) -- (6.5, 0);
        \end{tikzpicture}
        .
    \end{align}
    
    \chapter{Young Tableau}
    \section{What are Young Tableau}
    \begin{dfn}{Partition}{}
        Let \(k \in \naturals\).
        A \defineindex{partition} of \(k\) is a tuple, \(\tuple{\lambda_1, \dotsc, \lambda_n}\) such that \cite[720]{hassani}
        \begin{equation}
            k = \sum_{i = 1}^{n} \lambda_i.
        \end{equation}
        A partition is \define{ordered}\index{ordered partition} if \(\lambda_i \ge \lambda_{i+1}\) for all \(i = 1, \dotsc, n - 1\).
    \end{dfn}

    For example, \(\tuple{1,5,3}\) is a partition of 9, this is not ordered, the equivalent ordered partition is \(\tuple{5,3,1}\).
    
    \begin{dfn}{Young Diagram}{}
        Given an ordered partition \(\tuple{\lambda_1, \dotsc, \lambda_n}\) of some \(k \in \naturals\) the \defineindex{Young diagram} is formed from a row of \(\lambda_1\) boxes above a row of \(\lambda_2\) boxes and so on down to a row of \(\lambda_n\) boxes, all aligned to the left \cite[87]{cvitanovic}.
    \end{dfn}
    
    For example, the ordered partitions of four are \(\tuple{4}\), \(\tuple{3,1}\), \(\tuple{2,2}\), \(\tuple{2,1,1}\), and \(\tuple{1,1,1,1}\).
    The corresponding Young diagrams are
    \ytableausetup{smalltableaux}
    \begin{equation}
        \ydiagram{4}\,, \quad \ydiagram{3,1}\,, \quad \ydiagram{2,2}\,, \quad \ydiagram{2,1,1}\,, \qand \ydiagram{1,1,1,1}\,.
    \end{equation}
    
    \begin{dfn}{Young Tableau}{}
        Given a Young diagram we form a \defineindex{Young tableau} by writing numbers in the boxes in such a way that the numbers are increasing from left to right along a row and strictly increasing from top to bottom down a column \cite{cvitanovic}.
        A \(k\) box \defineindex{standard Young tableau} is a Young tableau using the numbers 1 through \(k\) exactly once.
    \end{dfn}
    
    For example, given the Young diagram associated with the partition \(\tuple{2,1,1}\) there are three distinct standard tableaux:
    \begin{equation}
        \ytableaushort{12,3,4}, \qquad \ytableaushort{13,2,4}, \qqand \ytableaushort{14,2,3}.
    \end{equation}
    
    \begin{dfn}{Hook Number}{}
        Given a Young diagram, \(Y\), the \defineindex{hook length} of a box is the number of boxes to the right of that box, plus the number of boxes below that box, plus 1 for the box itself.
        The \defineindex{hook number}, \(\hooknumber{Y}\), is the product of the hook lengths \cite[88]{cvitanovic}.
    \end{dfn}
    \begin{exm}{}{}
        Consider the Young diagram given by the partition \(\tuple{3,2}\):
        \begin{equation}
            Y = \ydiagram{3,2}.
        \end{equation}
        The hooks of this diagram are
        \begin{equation}
            \begin{ytableau}
                *(light highlight) \strut & *(light highlight) \strut & *(light highlight) \strut \\
                *(light highlight) \strut & \strut
            \end{ytableau}
            , \quad
            \begin{ytableau}
                \strut & *(light highlight) \strut & *(light highlight) \strut \\
                \strut & *(light highlight) \strut
            \end{ytableau}
            , \quad
            \begin{ytableau}
                \strut & \strut & *(light highlight) \strut \\
                \strut & \strut
            \end{ytableau}
            , \quad
            \begin{ytableau}
                \strut & \strut & \strut \\
                *(light highlight) \strut & *(light highlight) \strut
            \end{ytableau}
            , \qand
            \begin{ytableau}
                \strut & \strut & \strut \\
                \strut & *(light highlight) \strut
            \end{ytableau}
            .
        \end{equation}
        Labelling each box with the associated hook length gives
        \begin{equation}
            \ytableaushort{431,21}
        \end{equation}
        The hook number is
        \begin{equation}
            \hooknumber{Y} = 4 \cdot 3 \cdot 2 = 24.
        \end{equation}
    \end{exm}
    
    \section{Young Projectors}
    \begin{dfn}{Young Projector}{}
        Given a \(k\) box Young tableau, \(Y\), with \(r\) rows and \(c\) columns the \defineindex{Young projector}, \(\projector{Y}\) associated with \(Y\) is given by the following procedure \cite[91]{cvitanovic}:
        \begin{enumerate}
            \item Symmetrise over the numbers in each row.
            \item Antisymmetrise over the numbers in each column.
            \item Normalise by a factor of
            \begin{equation}
                \alpha_Y \coloneqq \frac{1}{\hooknumber{Y}} \left( \prod_{i=1}^{r} \abs{S_i}! \right) \left( \prod_{j=1}^{c} \abs{A_j}! \right).
            \end{equation}
            Here \(\hooknumber{Y}\) is the hook number of the diagram, \(\abs{S_i}\) is the length of the \(i\)th row and \(\abs{A_i}\) is the length of the \(j\)th column.
        \end{enumerate}
    \end{dfn}
    
    \begin{exm}{}{}
        Consider the \(5\) box Young tableau
        \begin{equation}
            Y = \ytableaushort{123,45}.
        \end{equation}
        Numbering the boxes we get
        \begin{equation}
            \ytableaushort{123,45}.
        \end{equation}
        We need to symmetrise over \(1\), \(2\) and \(3\), and \(4\) and \(5\), using \(S_{123}\) and \(S_{45}\), and antisymmetrise over \(1\) and \(4\), \(2\) and \(5\), and \(3\) on its own, using \(A_{14}\), \(A_{25}\), and \(A_{3} = \cycle{} = \identity\).
        
        The normalisation factor is
        \begin{align}
            \alpha_Y &= \frac{1}{24} (3!2!)(2!2!1!) = 6.
        \end{align}
        The Young projector is then
        \begin{equation}
            \projector{Y} = 6
            \tikzsetnextfilename{young-projector-example}
            \begin{tikzpicture}[baseline=(A)]
                \node (A) at (0, 0.9) {};
                \foreach \y in {0, 0.5, ..., 2} {
                    \draw[wire] (0, \y) -- ++ (0.5, 0);
                }
                \draw[wire, rounded corners] (1, 0) -- ++ (0.5, 0) -- ++ (0.5, 0.5) -- ++ (1.5, 0) -- ++ (0.5, -0.5) -- ++ (0.5, 0);
                \draw[wire, rounded corners] (1, 0.5) -- ++ (0.5, 0) -- ++ (0.5, 1) -- ++ (1.5, 0) -- ++ (0.5, -1) -- ++ (0.5, 0);
                \draw[underwire, rounded corners] (1, 1) -- ++ (0.5, 0) -- ++ (0.5, -1) -- ++ (1.5, 0) -- ++ (0.5, 1) -- ++ (0.5, 0);
                \draw[wire, rounded corners] (1, 1) -- ++ (0.5, 0) -- ++ (0.5, -1) -- ++ (1.5, 0) -- ++ (0.5, 1) -- ++ (0.5, 0);
                \draw[underwire, rounded corners] (1, 1.5) -- ++ (0.5, 0) -- ++ (0.5, -0.5) -- ++ (1.5, 0) -- ++ (0.5, 0.5) -- ++ (0.5, 0);
                \draw[wire, rounded corners] (1, 1.5) -- ++ (0.5, 0) -- ++ (0.5, -0.5) -- ++ (1.5, 0) -- ++ (0.5, 0.5) -- ++ (0.5, 0);
                \draw[wire, rounded corners] (1, 2) -- ++ (3.5, 0);
                
                \draw[symmetriser] (0.5, 2.15) rectangle (1, 0.85);
                \draw[symmetriser] (0.5, 0.65) rectangle (1, -0.15);
                \draw[antisymmetriser] (2.5, 2.15) rectangle (3, 1.35);
                \draw[antisymmetriser] (2.5, 1.15) rectangle (3, 0.35);
            \end{tikzpicture}
        \end{equation}
    \end{exm}
    
    % TODO: Citations for this section
    \section{Garnir Relations}
    The standard Young tableaux of a given shape, or rather the associated Young projectors, form a basis for the Young projectors formed from Young tableaux of that shape.
    The \defineindex{Garnir relations} give us a way of taking a projector corresponding to a nonstandard Young tableau and writing it as a sum of projectors corresponding to standard Young tableaux of the same shape.
    While the relations are on the level of the Young projectors, which are elements of the group algebra, \(\complex[\symmetricGroup]\), they are applied on the level of the Young tableaux.
    So just keep in mind that whenever we write a sum of two Young tableau the sum is actually happening one layer down in the group algebra.
    
    We start with a nonstandard \(k\) box Young tableau.
    The Garnir relations are best expressed through an example.
    We'll start with the \(k = 11\) nonstandard Young tableau
    \begin{equation}
        Y = \ytableaushort{1256,38{10},47,9{11}}.
    \end{equation}
    Since a Young projector involves symmetrising over a row the order of the numbers in a row is not important, so we can always sort them in increasing order.
    The numbers in each row are already sorted in increasing order here, but if they weren't then doing so would be the first step.
    The first step to applying the Garnir relations is to identify the first space, from top left going left to right top to bottom, where the tableau fails to be standard.
    In our example this is the 8 which is above a 7.
    We then highlight from this point to the point directly below it:
    \begin{equation}
        \begin{ytableau}
            1 & 2 & 5 & 6\\
            3 & *(light highlight) 8 & *(light highlight) 10\\
            *(light highlight) 4 & *(light highlight) 7\\
            9 & 11
        \end{ytableau}
    \end{equation}
    This gives us a strip of values, \(\tuple{8,10,4,7}\).
    However, we want to think of this strip not by the numbers in the boxes but the numbers of the boxes when labelled \(1\) to \(k\):
    \begin{equation}
        \ytableaushort{1234,567,89,{10}{11}}.
    \end{equation}
    So the boxes highlighted above are \(6\), \(7\), \(8\), and \(9\).
    
    The next step is to perform all permutations of this strip which give distinct Young projectors, which means that swaps within a row are all the same.
    In this case these swaps are
    \begin{equation}
        \cycle{6,8}, \quad \cycle{6,9}, \quad \cycle{7,8}, \quad \cycle{7,9}, \qand \cycle{6,8}\cycle{7,9}.
    \end{equation}
    These permutations act on the Young tableau by permuting the values in the boxes with these numbers.
    This gives us the tableaux
    \ytableausetup{smalltableaux}
    \begin{alignat}{3}
        \cycle{6,8} \action Y &=
        \ytableaushort{1256,34{10},87,9{11}}, \qquad &
        \cycle{6,8} \action Y &=
        \ytableaushort{1256,37{10},48,9{11}}, \qquad &
        \cycle{7,8} \action Y &=
        \ytableaushort{1256,384,{10}7,9{11}}, \notag\\
        \cycle{7,9} \action Y &=
        \ytableaushort{1256,387,4{10},9{11}}, \qquad &
        \cycle{6,8}\cycle{7,9} \action Y &=
        \ytableaushort{1256,347,8{10},9{11}}.
    \end{alignat}
    We can again sort all of the rows, giving
    \begin{alignat}{3}
        \cycle{6,8} \action Y &=
        \ytableaushort{1256,34{10},78,9{11}}, \qquad &
        \cycle{6,8} \action Y &=
        \ytableaushort{1256,37{10},48,9{11}}, \qquad &
        \cycle{7,8} \action Y &=
        \ytableaushort{1256,348,7{10},9{11}}, \notag\\
        \cycle{7,9} \action Y &=
        \ytableaushort{1256,378,4{10},9{11}}, \qquad &
        \cycle{6,8}\cycle{7,9} \action Y &=
        \ytableaushort{1256,347,8{10},9{11}}.
    \end{alignat}
    The Garnir relations then tell us that
    \begin{align}
        0 &= Y + \cycle{6,8} \action Y + \cycle{6,8} \action Y + \cycle{7,8} \action Y + \cycle{7,9} \action Y + \cycle{6,8}\cycle{7,9} \action Y\\
        &= \ytableaushort{1256,38{10},47,9{11}} + \ytableaushort{1256,34{10},78,9{11}} + \ytableaushort{1256,37{10},48,9{11}} + \ytableaushort{1256,348,7{10},9{11}} + \ytableaushort{1256,378,4{10},9{11}} + \ytableaushort{1256,347,8{10},9{11}}. \notag
    \end{align}
    Thus, we can write the initial nonstandard Young tableau as
    \begin{align}
        Y &= -\cycle{6,8} \action Y - \cycle{6,8} \action Y - \cycle{7,8} \action Y - \cycle{7,9} \action Y - \cycle{6,8}\cycle{7,9} \action Y\\
        &= -\ytableaushort{1256,34{10},78,9{11}} - \ytableaushort{1256,37{10},48,9{11}} - \ytableaushort{1256,348,7{10},9{11}} - \ytableaushort{1256,378,4{10},9{11}} - \ytableaushort{1256,347,8{10},9{11}}.
    \end{align}
    
    At this point there are two possibilities, if all of the Young tableaux appearing on the right are standard, as is the case here, then we are finished.
    If one or more is not standard then the first point at which it fails to be standard will have moved right and/or down, meaning it is, in a sense, closer to being standard.
    We can then apply the Garnir relations to this tableau and the failure point will once again move right and/or down.
    Since tableau are finite eventually this process must terminate and we will be left with only standard tableau.
    
    To summarise, the process of writing a nonstandard tableau in terms of standard tableau is as follows:
    \begin{enumerate}
        \item Sort the rows of the tableau into ascending order, if the result is a standard tableau stop here.
        \item Identify the first point from the top left going left to right and top to bottom where the tableau fails to be standard.
        \item Identify a strip going from this point to the box under this point.
        \item Apply all permutations to this strip which result in distinct tableau.
        \item The original tableau is then the negative of the sum of these permuted tableau.
        \item If any of the tableau in this sum are nonstandard recursively apply the Garnir relations.
        When no nonstandard tableau remain stop.
    \end{enumerate}
    Another example, where the Garnir relations are applied recursively, is in \cref{app:garnir example}
    
    \section{Irreducible Representations of The Symmetric Group}
    Given a Young tableau, for example
    \begin{equation}
        Y = \ytableaushort{124,35},
    \end{equation}
    we can identify a permutation, \(\sigma\), which acts on the Young tableau of the same shape but with all numbers in order from left to right top to bottom, which we'll call the \defineindex{ordered tableau} of this shape, by permuting the boxes based on their numbers to produce \(Y\), that is
    \begin{equation}
        \ytableaushort{124,35} = \sigma \action \ytableaushort{123,45},
    \end{equation}
    In our example \(\sigma = \cycle{3,4}\).
    In this way we get a one-to-one correspondence between \(n\) box Young tableaux and permutations in \(\symmetricGroup\).
    
    Representations of \(\symmetricGroup\) can be labelled by ordered partitions of \(n\).
    This is because for a finite group there is a one-to-one correspondence between irreducible representations and conjugacy classes \cite{zhenheng}, and the conjugacy classes are grouped by cycle type \cite{conjugacy-classes-cycle-types}, which is simply the length of the cycles, for example \(\cycle{1,2,3}\cycle{4,5}\) has cycle type \(\tuple{3,2}\) in \(\symmetricGroup[5]\), or viewed in \(\symmetricGroup[7]\) we can write it as \(\cycle{1,2,3}\cycle{4,5}\cycle{6}\cycle{7}\), in which case it has cycle type \(\tuple{3,2,1,1}\).
    So, we have a one-to-one correspondence between partitions and conjugacy classes, a one-to-one correspondence between conjugacy classes and representations, and a one-to-one correspondence between partitions and Young diagrams.
    Hence, there is a one-to-one correspondence between \(n\) box Young diagrams and representations of \(\symmetricGroup\) allowing us to label representations of \(\symmetricGroup\) by \(n\) box Young diagrams.
    This is summarised in \cref{fig:one-to-one relations in Sn}
    
    \begin{figure}
        \tikzexternaldisable
        \begin{tikzcd}
            \text{irreps of \(\symmetricGroup\)} \arrow[r, <->] \arrow[d, <->] & \text{conjugacy classes of \(\symmetricGroup\)} \arrow[d, <->]\\
            \text{\(n\) box Young diagrams} \arrow[r, <->] & \text{partitions of \(n\)}
        \end{tikzcd}
        \begin{tikzcd}
            \text{elements of \(\symmetricGroup\)} \arrow[r, <->] & \text{\(n\) box Young tableau of a given shape}
        \end{tikzcd}
        \tikzexternalenable
        \caption{One-to-one relations between various concepts relating to Young tableaux and representations of \(S_n\).}
        \label{fig:one-to-one relations in Sn}
    \end{figure}
    
    The question now is given some \(\sigma \in \symmetricGroup\) and some \(n\) box Young diagram, \(Y\), how do we find the matrix corresponding to \(\sigma\) in the representation labelled by \(Y\)?
    We'll follow demonstrate the procedure with the example \(\sigma = \cycle{1,3}\) and the representation labelled by the Young diagram
    \begin{equation}
        \ydiagram{2,1}.
    \end{equation}
    Start by identifying this Young diagram with the ordered Young tableau of the same shape:
    \begin{equation}
        Y = \ytableaushort{12,3}.
    \end{equation}
    The dimension of the representation is given by the number of standard tableaux of the same shape as \(Y\), in this example there are two:
    \begin{equation}
        \ytableaushort{12,3}, \qqand \ytableaushort{13,2}.
    \end{equation}
    From each of these we can identify a standard permutation, the permutation giving the standard tableau from the ordered tableau, in this case the standard permutations are \(\cycle{}\) and \(\cycle{2,3}\).
    We need the Young projector associated with \(Y\):
    \begin{equation}
        \projector{Y} = \frac{4}{3} \,
        \tikzsetnextfilename{rep-of-Sn-example-Young-projector}
        \begin{tikzpicture}[baseline=0.42cm]
            \draw[wire] (0, 1) -- (0.5, 1);
            \draw[wire] (0, 0.5) -- (0.5, 0.5);
            \draw[wire] (1, 1) -- (2.5, 1);
            \draw[wire] (3, 1) -- (4.5, 1);
            \draw[wire, rounded corners] (0, 0) -- (1.5, 0) --  (2, 0.5) -- (2.5, 0.5);
            \draw[wire, rounded corners] (3, 0.5) -- (3.5, 0.5) -- (4, 0) -- (4.5, 0);
            \draw[underwire, rounded corners] (1, 0.5) -- (1.5, 0.5) -- (2, 0) -- (3.5, 0) -- (4, 0.5) -- (4.5, 0.5);
            \draw[wire, rounded corners] (1, 0.5) -- (1.5, 0.5) -- (2, 0) -- (3.5, 0) -- (4, 0.5) -- (4.5, 0.5);
            \draw[symmetriser] (0.5, 0.25) rectangle (1, 1.25);
            \draw[antisymmetriser] (2.5, 0.25) rectangle (3, 1.25);
        \end{tikzpicture}
        =
        \tikzsetnextfilename{rep-ofSn-example-Young-projector-short-form}
        \begin{tikzpicture}[baseline=0.42cm]
            \draw[wire] (0, 1) -- (0.5, 1);
            \draw[wire] (0, 0.5) -- (0.5, 0.5);
            \draw[wire] (0, 0) -- (0.5, 0);
            \draw[wire] (1.5, 1) -- (2, 1);
            \draw[wire] (1.5, 0.5) -- (2, 0.5);
            \draw[wire] (1.5, 0) -- (2, 0);
            \draw (0.5, -0.25) rectangle (1.5, 1.25);
            \node at (1, 0.5) {\ytableaushort{12,3}};
        \end{tikzpicture}
        .
    \end{equation}
    If we act on this with our two standard permutations we get
    \begin{equation}
        \cycle{} \action \projector{Y} = \projector{Y} = \ve{1}, \qqand \cycle{2,3} \action \projector{Y} = 
        \tikzsetnextfilename{rep-of-Sn-second-basis-element}
        \begin{tikzpicture}[baseline=0.42cm]
            \draw[wire] (-0.5, 1) -- (0, 1) -- (0.5, 1);
            \draw[wire, rounded corners] (-0.5, 0) -- (-0.25, 0) -- (0.25, 0.5) -- (0.5, 0.5);
            \draw[underwire, rounded corners] (-0.5, 0.5) -- (-0.25, 0.5) -- (0.25, 0) -- (0.5, 0);
            \draw[wire, rounded corners] (-0.5, 0.5) -- (-0.25, 0.5) -- (0.25, 0) -- (0.5, 0);
            \draw[wire] (1.5, 1) -- (2, 1);
            \draw[wire] (1.5, 0.5) -- (2, 0.5);
            \draw[wire] (1.5, 0) -- (2, 0);
            \draw (0.5, -0.25) rectangle (1.5, 1.25);
            \node at (1, 0.5) {\ytableaushort{12,3}};
        \end{tikzpicture}
        = \ve{2}.
    \end{equation}
    We take these two projectors to be the basis vectors in our representation space.
    We then act on them with \(\sigma = \cycle{1,3}\), giving
    \begin{align}
        \cycle{1,3} \action \ve{1} &=
        \tikzsetnextfilename{rep-of-Sn-step-1}
        \begin{tikzpicture}[baseline=0.42cm]
            \draw[wire, rounded corners] (-0.5, 1) -- (-0.25, 1) -- (0.25, 0) -- (0.5, 0);
            \draw[underwire] (-0.5, 0.5) -- (0.5, 0.5);
            \draw[wire] (-0.5, 0.5) -- (0.5, 0.5);
            \draw[underwire, rounded corners] (-0.5, 0) -- (-0.25, 0) -- (0.25, 1) -- (0.5, 1);
            \draw[wire, rounded corners] (-0.5, 0) -- (-0.25, 0) -- (0.25, 1) -- (0.5, 1);
            \draw[wire] (1.5, 1) -- (2, 1);
            \draw[wire] (1.5, 0.5) -- (2, 0.5);
            \draw[wire] (1.5, 0) -- (2, 0);
            \draw (0.5, -0.25) rectangle (1.5, 1.25);
            \node at (1, 0.5) {\ytableaushort{12,3}};
        \end{tikzpicture}
        =
        \tikzsetnextfilename{rep-of-Sn-step-2}
        \begin{tikzpicture}[baseline=0.42cm]
            \draw[wire] (0, 1) -- (0.5, 1);
            \draw[wire] (0, 0.5) -- (0.5, 0.5);
            \draw[wire] (0, 0) -- (0.5, 0);
            \draw[wire] (1.5, 1) -- (2, 1);
            \draw[wire] (1.5, 0.5) -- (2, 0.5);
            \draw[wire] (1.5, 0) -- (2, 0);
            \draw (0.5, -0.25) rectangle (1.5, 1.25);
            \node at (1, 0.5) {\ytableaushort{32,1}};
        \end{tikzpicture}
        =
        \tikzsetnextfilename{rep-of-Sn-step-3}
        \begin{tikzpicture}[baseline=0.42cm]
            \draw[wire] (0, 1) -- (0.5, 1);
            \draw[wire] (0, 0.5) -- (0.5, 0.5);
            \draw[wire] (0, 0) -- (0.5, 0);
            \draw[wire] (2.5, 1) -- (3, 1);
            \draw[wire] (2.5, 0.5) -- (3, 0.5);
            \draw[wire] (2.5, 0) -- (3, 0);
            \draw (0.5, -0.25) rectangle (2.5, 1.25);
            \node at (1.5, 0.5) {-\,\ytableaushort{12,3} \, - \ytableaushort{13,2}};
        \end{tikzpicture}
        \notag\\
        &= -
        \tikzsetnextfilename{rep-of-Sn-step-4}
        \begin{tikzpicture}[baseline=0.42cm]
            \draw[wire] (0, 1) -- (0.5, 1);
            \draw[wire] (0, 0.5) -- (0.5, 0.5);
            \draw[wire] (0, 0) -- (0.5, 0);
            \draw[wire] (1.5, 1) -- (2, 1);
            \draw[wire] (1.5, 0.5) -- (2, 0.5);
            \draw[wire] (1.5, 0) -- (2, 0);
            \draw (0.5, -0.25) rectangle (1.5, 1.25);
            \node at (1, 0.5) {\ytableaushort{12,3}};
        \end{tikzpicture}
        -
        \tikzsetnextfilename{rep-of-Sn-step-5}
        \begin{tikzpicture}[baseline=0.42cm]
            \draw[wire] (-0.5, 1) -- (0, 1) -- (0.5, 1);
            \draw[wire, rounded corners] (-0.5, 0) -- (-0.25, 0) -- (0.25, 0.5) -- (0.5, 0.5);
            \draw[underwire, rounded corners] (-0.5, 0.5) -- (-0.25, 0.5) -- (0.25, 0) -- (0.5, 0);
            \draw[wire, rounded corners] (-0.5, 0.5) -- (-0.25, 0.5) -- (0.25, 0) -- (0.5, 0);
            \draw[wire] (1.5, 1) -- (2, 1);
            \draw[wire] (1.5, 0.5) -- (2, 0.5);
            \draw[wire] (1.5, 0) -- (2, 0);
            \draw (0.5, -0.25) rectangle (1.5, 1.25);
            \node at (1, 0.5) {\ytableaushort{12,3}};
        \end{tikzpicture}
        = -\ve{1} - \ve{2}.
    \end{align}
    Here we act on the inputs to the Young projector, then interpret this as acting on the underlying Young tableau, which we can then decompose into a sum of standard Young tableaux using the Garnir relations, and then we can write each standard tableau as a permutation acting on the Young projector corresponding to the ordered Young tableau.
    We can do exactly the same with the second basis vector to get
    \begin{align}
        \cycle{1,3} \action \ve{2} &= 
        \tikzsetnextfilename{rep-of-Sn-step-6}
        \begin{tikzpicture}[baseline=0.42cm]
            \draw[wire, rounded corners] (-1.5, 1) -- (-1.25, 1) -- (-0.75, 0) -- (-0.5, 0) -- (-0.25, 0) -- (0.25, 0.5) -- (0.5, 0.5);
            \draw[underwire, rounded corners] (-1.5, 0.5) -- (-0.5, 0.5) -- (-0.25, 0.5) -- (0.25, 0) -- (0.5, 0);
            \draw[wire, rounded corners] (-1.5, 0.5) -- (-0.5, 0.5) -- (-0.25, 0.5) -- (0.25, 0) -- (0.5, 0);
            \draw[underwire, rounded corners] (-1.5, 0) -- (-1.25, 0) -- (-0.75, 1) -- (0.5, 1);
            \draw[wire, rounded corners] (-1.5, 0) -- (-1.25, 0) -- (-0.75, 1) -- (0.5, 1);
            \draw[wire] (1.5, 1) -- (2, 1);
            \draw[wire] (1.5, 0.5) -- (2, 0.5);
            \draw[wire] (1.5, 0) -- (2, 0);
            \draw (0.5, -0.25) rectangle (1.5, 1.25);
            \node at (1, 0.5) {\ytableaushort{12,3}};
        \end{tikzpicture}
        =
        \tikzsetnextfilename{rep-of-Sn-step-7}
        \begin{tikzpicture}[baseline=0.42cm]
            \draw[wire, rounded corners] (-0.5, 0) -- (-0.25, 0) -- (0.25, 1) -- (0.5, 1);
            \draw[underwire, rounded corners] (-0.5, 0.5) -- (-0.25, 0.5) -- (0.25, 0) -- (0.5, 0);
            \draw[underwire, rounded corners] (-0.5, 1) -- (-0.25, 1) -- (0.25, 0.5) -- (0.5, 0.5);
            \draw[wire, rounded corners] (-0.5, 0.5) -- (-0.25, 0.5) -- (0.25, 0) -- (0.5, 0);
            \draw[wire, rounded corners] (-0.5, 1) -- (-0.25, 1) -- (0.25, 0.5) -- (0.5, 0.5);
            \draw[wire] (1.5, 1) -- (2, 1);
            \draw[wire] (1.5, 0.5) -- (2, 0.5);
            \draw[wire] (1.5, 0) -- (2, 0);
            \draw (0.5, -0.25) rectangle (1.5, 1.25);
            \node at (1, 0.5) {\ytableaushort{12,3}};
        \end{tikzpicture}
        \\
        &=
        \tikzsetnextfilename{rep-of-Sn-step-8}
        \begin{tikzpicture}[baseline=0.42cm]
            \draw[wire] (-0.5, 1) -- (0, 1) -- (0.5, 1);
            \draw[wire, rounded corners] (-0.5, 0) -- (-0.25, 0) -- (0.25, 0.5) -- (0.5, 0.5);
            \draw[underwire, rounded corners] (-0.5, 0.5) -- (-0.25, 0.5) -- (0.25, 0) -- (0.5, 0);
            \draw[wire, rounded corners] (-0.5, 0.5) -- (-0.25, 0.5) -- (0.25, 0) -- (0.5, 0);
            \draw[wire] (1.5, 1) -- (2, 1);
            \draw[wire] (1.5, 0.5) -- (2, 0.5);
            \draw[wire] (1.5, 0) -- (2, 0);
            \draw (0.5, -0.25) rectangle (1.5, 1.25);
            \node at (1, 0.5) {\ytableaushort{12,3}};
        \end{tikzpicture}
        = \ve{2}.
    \end{align}
    In the last step we identified that the projector is symmetric in the first two inputs and so we are free to swap the first two input wires, which leaves us with \(\ve{2}\).
    If we hadn't seen that this was possible then we could have applied the permutation to the Young tableau and then decomposed the result with the Garnir relations, and the answer would be the same, it's just slower.
    We can then use the action on these two basis vectors to give the action on the entire representation space by considering the equation
    \begin{equation}
        \begin{pmatrix}
            \ve{1} & \ve{2}
        \end{pmatrix}
        \rho\cycle{1,3} = 
        \begin{pmatrix}
            -\ve{1} - \ve{2} & \ve{2}
        \end{pmatrix}
        \implies \rho\cycle{1,3} = 
        \begin{pmatrix}
            -1 & 0\\
            -1 & 1
        \end{pmatrix}
        .
    \end{equation}
    To summarise the process given some \(\sigma \in \symmetricGroup\) and an \(n\) box Young diagram labelling a representation we can find the action of \(\sigma\) in this representation as follows:
    \begin{enumerate}
        \item Write down the \(d\) standard tableaux of the same shape, these give the standard permutations, \(\tau_i\).
        \item Act on the ordered Young projector with the standard permutations to get the basis elements, \(\ve{i} = \tau_i \action \projector{Y}\).
        \item Act on each basis element with the permutation, \(\sigma \action \ve{i}\).
        \item Simplify the result using Garnir relations until the result is written in terms of \(\ve{i}\).
        \item Solve the equation
        \begin{equation}
            \begin{pmatrix}
                \ve{1} & \cdots & \ve{d}
            \end{pmatrix}
            \rho(\sigma) = 
            \begin{pmatrix}
                \sigma \action \ve{1} & \cdots & \sigma \action \ve{d}
            \end{pmatrix}
        \end{equation}
        for the \(d \times d\) representation matrix, \(\rho(\sigma)\).
    \end{enumerate}
    
    
    
    % Appdendix
    \appendixpage
    \begin{appendices}
        \chapter{Garnir Relations Example}\label{app:garnir example}
        \section{Recursive Application}
        Consider the Young tableau
        \begin{equation}
            Y = \ytableaushort{14,35,2}.
        \end{equation}
        The rows are already sorted.
        The problem strip starts with the 3:
        \begin{equation}
            \begin{ytableau}
                1 & 4\\
                *(light highlight) 3 & *(light highlight) 5\\
                *(light highlight) 2
            \end{ytableau}
        \end{equation}
        The box numbers are
        \begin{equation}
            \ytableaushort{12,34,5},
        \end{equation}
        so the boxes in the strip are numbers 3, 4, and 5.
        We need to consider the permutations \(\cycle{3,5}\) and \(\cycle{4,5}\), giving
        \begin{equation}
            \cycle{3,5} \action Y = \ytableaushort{14,25,3}, \qqand \cycle{4,5} \action Y = \ytableaushort{14,32,5}.
        \end{equation}
        Sorting the rows we get the tableaux
        \begin{equation}
            \ytableaushort{14,25,3}, \qqand \ytableaushort{14,23,5}.
        \end{equation}
        Thus
        \begin{equation}
            Y = \ytableaushort{14,35,2} = -\ytableaushort{14,25,3} - \ytableaushort{14,23,5}.
        \end{equation}
        The first tableau is standard, but the second isn't, we have a four above a 3.
        So, we need to recursively apply the Garnir relations.
        
        Starting again with
        \begin{equation}
            \tilde{Y} = \ytableaushort{14,23,5}
        \end{equation}
        we identify the problem strip:
        \begin{equation}
            \begin{ytableau}
                1 & *(light highlight) 4\\
                *(light highlight) 2 & *(light highlight) 3\\
                5
            \end{ytableau}
        \end{equation}
        Notice that the problem point has moved right compared to the starting tableau \(Y\).
        We need to consider the permutations \(\cycle{2,3}\) and \(\cycle{2,4}\).
        Applying these gives
        \begin{equation}
            \cycle{2,3} \action \tilde{Y} = \ytableaushort{12,43,5}, \qqand \cycle{2,4} \action \tilde{Y} = \ytableaushort{13,24,5}.
        \end{equation}
        Sorting the rows gives
        \begin{equation}
            \ytableaushort{12,34,5}, \qqand \ytableaushort{13,24,5}.
        \end{equation}
        Hence, we have
        \begin{equation}
            \tilde{Y} = -\ytableaushort{12,34,5} - \ytableaushort{13,24,5}.
        \end{equation}
        
        Going back to our original tableau we can express \(Y\) as a sum of standard tableaux:
        \begin{align}
            Y &= - \ytableaushort{14,25,3} - \ytableaushort{14,23,5}\\
            &= -\ytableaushort{14,25,3} - \left( -\ytableaushort{12,34,5} - \ytableaushort{13,24,5} \, \right)\\
            &= -\ytableaushort{14,25,3} + \ytableaushort{12,34,5} + \ytableaushort{13,24,5}.
        \end{align}
        Notice that since the Young projectors, denoted here by the corresponding Young tableaux, are elements of an algenbra, \(\complex[\symmetricGroup[5]]\), we can do all of the expected algebra with them such as distributing scalars across brackets since \(\complex[\symmetricGroup[5]]\) is a vector space.
    \end{appendices}

    \backmatter
    \printbibliography
    %    \renewcommand{\glossaryname}{Acronyms}
    %    \printglossary[acronym]
    \printindex
\end{document}